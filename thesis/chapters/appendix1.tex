\section{Exact Diagonalization}\label{sec:ed_imp}
The details to implement exact diagonalization for the canonical ensemble (CE) is described in great detail in Zhang et. al. (2010)\cite{Zhang_2010}. We will instead discuss a scheme to extend it to the grand canonical ensemble (GCE) as is required to implement the cluster mean field approximation.
\vspace{0.5cm}\\
Once we have written a function to compute the Hamiltonian $H(N, L)$ for a system of $N$ bosons on $L$ lattice sites, the GCE Hamiltonian is simply given by the direct sum $H_{\text{GCE}} = \oplus_{N = 1}^{\infty} H(N, L)$. For numerical feasibility, we will set an upper bound on the particle number, $N \leq N_{max}$. This effectively means that $H_{GCE}$ can be constructed as a block diagonal matrix using the set of CE Hamiltonians, $\{H(N, L)\}|_{N=1}^{N_{max}}$, since $\hat{H}_{BHM}$ commutes with $\hat{N} = \sum_{i} \hat{n}_i$. While the scheme outlined above is perfectly valid, we will also describe a different approach due to its similarity with the implementation rrquired for the mean-field approximation. 
\vspace{0.5cm}\\
Consider the local annihilation operator, $\tilde a_i$, on a particlar site $i$. Given a maximum particle occupation of $N_{max}$, we can write the operator in the local occupation basis $\{n_i\}$ as a sparse matrix with $\{ \sqrt{1}, \sqrt{2}, \dots, \sqrt{N_{max}}\}$ as the lower off-diagonal. We can then construct $\tilde a_i^{\dagger}$ as the conjugate transpose and $\tilde n_i$ as $\tilde a_i^{\dagger} \tilde a_i$. Note that the bosonic commutation relations imply a tensor product structure for the combined space of multiple bosonic particles (which is not the case for fermions). As a result, we can construct the global/lattice operators, $O_i$ using the local operators $\tilde O_i$ as follows.
\begin{equation}
    O_i = \underbrace{\mathbb{I} \otimes \dots \otimes \mathbb{I}}_{(i-1)\text{ times}} \otimes \tilde O_i \otimes \underbrace{\mathbb{I} \otimes \dots \otimes \mathbb{I}}_{(L - i) \text{ times}} 
\end{equation}
The GCE Hamiltonian can then be easily constructed using the constituent lattice operators $a_i$, $a_i^{\dagger}$ and $n_i$. 
\vspace{0.5cm}\\
A closer look reveals that the two schemes differ only in the choice of basis that we have utilized to construct the Hamiltonian. The first case uses a direct sum of basis sets of each fixed particle number subspace, $\ket{\Psi} = \oplus_{N=1}^{N_{max}}\ket{\Psi(N, L)}$, whereas the second case uses a tensor product of the local occupation basis, $\ket{\Psi'} = \otimes_{k=1}^{L}\ket{\Psi'(N_{max}, k)}$.

\section{Mean field theory}

\subsection{Bose Hubbard model}
In this case, we simply have to construct the single-site Hamiltonian in Eq. \eqref{eq:bhm_mft} given a maximum particle occupation of $N_{max}$. Since we are forced to work with the GCE within this mean-field decoupling, we note that we have already constructed the required local site operators as discussed in Sec. \ref{sec:ed_imp}. The local Hamiltonian is then easily constructed and the mean-field parameter is determined self-consistently using fixed point iteration. A combination of absolute and relative tolerances are used to set the convergence limit for the self-consistency loop, i.e, the condition is as follows; $(x^{(n + 1)} - x^{(n)})  \leq \epsilon_{\text{atol}} + \epsilon_{\text{rtol}} * x^{(n)}$. 

\subsection{Extended Bose Hubbard model}
In this case, we have to construct the Hamiltonian over a unit cell of lattice sites as shown in Eq. \eqref{eq:ebhm_mft}. The unit cell operators can be constructed using the tensor product structure discussed in Sec. \ref{sec:ed_imp} except that we use the number of sites in the unit cell, $L_{UC}$, instead of the total number of lattice sites, $L$. The Hamiltonian can then be constructed for an arbitrary unit cell by taking the connectivity matrix as an input and using the relation in Eq. \eqref{eq:unit_cell} to build the mean-field decoupled terms. The non-linearity introduced by the unit cell structure result in convergence issues during the self-consistency procedure. This has been discussed in depth in Sec. \ref{sec:caveats}.

\subsection{Spin-1 Bose Hubbard model}
This case is quite similar to the Bose Hubbard model, however, we must construct three kinds of creation/annihilation operators corresponding to each spin projection, $\sigma \in \{1, 0, \overline{1}\}$. Note that we have an important constraint to consider, namely, that $\sum_{\sigma} n_{i\sigma} \leq N_{max}$ on each lattice site. This can be enforced by enumerating the Fock-space basis set for a lattice with 3 sites and $N$ bosons such that $N \leq N_{max}$. The procedure is the same as the one used in constructing the CE Hamiltonian for exact diagonalization in Sec. \ref{sec:ed_imp}.
\vspace{0.5cm}\\
While it is tempting instead to utilize a tensor product structure as in the previous section, it becomes cumbersome since we require $\sum_{\sigma} n_{i\sigma} \leq N_{max}$ instead of the naturally imposed condition, $n_{i\sigma} \leq N_{max}$. The second case would effectively result in a maximum site occupation of $3N_{max}$, however, it only takes into account the lower energy states since each spin state occupation cannot exceed $N_{max}$. As a result, the tensor product technique would not enumerate the entire basis set consistently. Surprisingly, the phase boundary predicted by this method still matches the true result, although the nature of the phases are altered (for e.g. the net spin in Mott insulator lobes are capped at $N_{max}$ instead of $3N_{max}$).

\section{Code repository}

All code used in this project was written in \href{https://julialang.org/}{Julia 1.8}\cite{Julia-2017} and can be found at \url{https://github.com/20akshay00/MSThesis}. All figures and diagrams were generated using \href{https://github.com/JuliaPlots/Plots.jl}{Plots.jl}\cite{christ2022plotsjl} and \href{https://github.com/JuliaGraphics/Luxor.jl}{Luxor.jl}. The following packages were utilized in varying capacities for implementing the numerical techniques: \href{https://github.com/JuliaNLSolvers/Optim.jl}{Optim.jl}\cite{mogensen2018optim}, \href{https://github.com/francescoalemanno/FixedPoint.jl}{FixedPoint.jl} and \href{https://github.com/Jutho/KrylovKit.jl}{KrylovKit.jl}. 