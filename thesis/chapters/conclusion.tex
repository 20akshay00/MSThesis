We started by setting up the formalism to describe the physics of interacting bosons in a periodic lattice. We then proceeded to generate the ground state phase diagram for bosons with contact interactions and successfully extracted the phase boundary for the Mott insulator to superfluid transition. This was done numerically using Mean-field and Cluster Mean-field techniques.
\vspace{0.5cm}\\
Upon deeming the latter to be impractical for our requirements, we moved on to study the ground state phases exhibited in the presence of long-range interactions at a mean-field level. As a result, we observed extended regions of two new phases, density waves and supersolids for any non-zero magnitude of the strength of nearest-neighbour interactions. We then introduced a spin-degree of freedom and analyzed its effect on the nature of the Mott insulator to superfluid transition. This directly led to a simple analysis of the phenomenon of mediation through bosons to induce effective interactions between non-interacting particles.
\vspace{0.5cm}\\
Finally, we presented a brief review of Variational QMC and Stochastic Series Expansion to study the Bose-Hubbard model beyond the mean-field level. While the basic framework was implemented and initial results were obtained, this line of work is still at an early stage and did not generate tangible insight. In the future, we plan to corroborate and extend the results obtained in this thesis using the state-of-the-art worm algorithm to study the finite temperature physics of the Bose-Hubbard model.

